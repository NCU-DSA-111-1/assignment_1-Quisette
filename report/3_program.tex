	\section{Program}
	This section contains the introduction of some parameters and self-defined functions to make users feel free reading source code. 
	\subsection{Macro-Defined Constants}
	\begin{table}[ht]
	    \centering
	    \begin{tabular}{|l|l|l|}
	    \hline
	        Name  &  Description & Default Value\\
	         \hline
    	     ANN\_INPUT\_NUM &  numbers of inputs in NN model &2\\ 
    	     ANN\_HIDDEN\_LAYERS & numbers of hidden layers in NN  model &1\\ 
    	     ANN\_NEURONS\_PER\_LAYER &  neurons per layer in NN model &2\\ 
    	     ANN\_OUTPUT\_NUM & numbers of inputs in NN model &1\\ 
    	     ANN\_LEARNING\_RATE & the learning rate of neural network &3\\ 
    	     LOOKUP\_SIZE & temporary data storage for \textit{genann}  &4096\\ 
    	     MAX\_CHAR\_LEN & maximum limit of user string input &100\\ 
    	     ZERO\_ASCII & value of 0 in ASCII &48\\ 
    	     TRAIN\_SETS & sets of neural network training
    	     &4\\ 
    	      \hline
	    \end{tabular}
	    \caption{Table of Macro-Defined Constants}
	    \label{tab:my_label}
	\end{table}
	\subsection{Self-Defined Functions}
	\paragraph{void init();\\}
	This function initializes the space of the neural network's lookup size and array sample I/Os for later usage. 
	\paragraph{int selectMode();\\}
	This function asks the user to switch between functions by entering numbers corresponding to the functions by the instruction.
	The function will return an integer representing the enumeration code of functions.   
	\paragraph{char *getInput();\\}
	This function works when the user is asked to enter a string to show its checksum. The function will return the address of the stored string.
	\paragraph{void nnQuadraticLoss(int iteration, int lossReprtSteps);\\}
	This function is used to train the model and display loss function (MSE) based on the given parameters:
	\begin{itemize}
	    \item \lstinline{int iteration} : the number of iteration times to train this model.
	    \item \lstinline{int lossReportSteps} : the interval of generating a report (a line of output in console) of loss. 
	\end{itemize}
	For instance, if we use  \lstinline{nnQuadraticLoss(30000,1000);} , we will get  30000 times of model training and a report in every 1000 epochs.
	\paragraph{void nnTrain(int iteration);\\}
	This function trains the model without showing the loss function. 
	\paragraph{void getBitString(char *string);\\}
	This function generates a string of bits based on the  string user gives. The function will generate a string of bits and call the following function:
	\begin{itemize}
	    \item \textbf{void getRealAns(char *string);} prints the binary parity checksum based on the pre-built XOR operator on the standard library. 
	    \item \textbf{void getTrainAns(char *string);} prints the binary parity checksum based on learned results.  
	\end{itemize}
	
	
	
	\paragraph{void calloc2dArray(double **array, int row, int col);\\}
	This function creates a 2-dimensional array based on the given parameters:
	\begin{itemize}
	    \item \lstinline{double **array} : the address of the 2-dimensional array.
	    \item \lstinline{int row, int col} : the rows and columns of the array. 
	\end{itemize}
	\paragraph{bool nnXor(char num1, char num2);\\}
	    This function takes two characters (only 0 and 1 are allowed) and maps the inputs to the corresponding outputs from learned results.

	\subsection{Compile Parameters}
	    Please open the project directory in command line and execute the following command:
	    \begin{lstlisting}[language=Bash,caption=Compile Parameters]
$ make main 
$ ./main	     \end{lstlisting}
	    \begin{lstlisting}[language=Bash,caption=Compile Results]
cc -Wall -Wshadow -O3 -g -march=native   -c -o main.o main.c
cc   main.o genann.o commonFunctions.o  -lm -o main	     \end{lstlisting}
Note that the warnings from the compiler has been omitted for concise. 