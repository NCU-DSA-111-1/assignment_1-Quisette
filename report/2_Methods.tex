	\section{Techniques and Resources}
	In this section, we want to introduce a useful concept called \emph{dynamic memory allocation}, and compare it with the known static memory allocation. We also mention the function library \emph{genann} to use, for its simplicity to include. 
 	\subsection{Dynamic Memory Allocation in C}
	    \subsubsection{Difference of Static and Dynamic Memory Allocation }
	    According to reference \cite{ref1}, the author has concluded the advantages and disadvantages of static memory allocation and dynamic memory allocation, shown below:
	    \paragraph{Static Memory Allocation}
	    \begin{itemize}
	        \item Allocation of memory is done automatically by the compiler. This will reduce the execution time of the program.
	        \item Uses stack data structures.
	        \item Allocated memory (including variables and arrays) cannot be freed until the program terminates.
	        \item Memory is allocated before the script executes.
	        \item Simple usage and easy for new learners to use.
	    \end{itemize}
	    \paragraph{Dynamic Memory Allocation}
	    \begin{itemize}
	        \item Memory is allocated in runtime. Hence, the execution time will be increased.
	        \item Uses heap data structures.
	        \item Memory can be resized/freed by scripts, therefore lessening wasted memory in the program.
	        \item Memory should be freed manually after use, or it could cause unknown problems to debug.  
	        \item Needs better knowledge of memory allocation to prevent unknown errors.
	        
	    \end{itemize}
	   
	    \subsubsection{Usable Functions to Dynamically Allocate Memory}
	    \begin{itemize}
	        \item \lstinline{malloc()} allocates the dynamic memory without the initialization. 
	        \item \lstinline{calloc()} allocates the dynamic memory with initialization to prevent accidentally accessing  corrupted memory. 
	        \item \lstinline{realloc()} reallocates the existing memory to expand/shrink its memory size.
	        \item \lstinline{free()} frees the memory and let it be recycled by the operating system.
	    \end{itemize}
	     
	    \subsection{Function Libraries Used in This Program}
	    In this program, we used an open-sourced neural network library Genann \cite{ref2}, with a small number of function library file to include and clear documentation for users to embed into their codes.    
